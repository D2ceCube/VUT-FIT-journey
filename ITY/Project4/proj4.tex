\documentclass[a4paper, 11pt]{article}
\usepackage[left=2cm, top=3cm, text={17cm, 24cm}]{geometry}
\usepackage[czech]{babel}
\usepackage[utf8]{inputenc}
\usepackage[unicode]{hyperref}
\usepackage{times}



\begin{document}
\begin{titlepage}
        \begin{center}
            \textsc{\Huge ~Vysoké učení technické v Brně \\[0.3em]
            \huge{Fakulta informačních technologií}}\\
            \vspace{\stretch{0.382}}
            {\LARGE Typografie a publikování\,--\,4. projekt\\[0.3em]
            \Huge{Typografie}}
            \vspace{\stretch{0.618}}
        \end{center}
        {\Large\today \hfill Assatulla Dias }
    \end{titlepage}
    \newpage
    
    \section{Definice}
    Typografie\,--\,je umění navrhování tištěného textu, založeného na určitých pravidlech, která jsou vlastní konkrétnímu jazyku, prostřednictvím psaní a rozvržení.
    Více o tom, proč je typografie důležitá, si můžete přečíst v tomto zdroji.\cite{Danielle2013}
    
    \section{Historie}
    
    \subsection{Počátek}
    Typografii s pohyblivým písmem vynalezl během dynastie Song v Číně v jedenáctém století Bi Sheng'em. Jeho typový systém byl vyroben z keramických materiálů a hliněný tisk pokračoval v Číně až do dynastie Čching. Více podrobností můžete nalezt v této knize.\cite{Nedham1994}
    
    \subsection{Dějiny typografie v Evropě}
    Tradičně je výchozím bodem pro dějiny tisku v Evropě Gutenbergova bible, vydání Vulgáty vydané Johannesem Gutenbergem v první polovině 50. let 14. století. Přestože se nejedná o první prvotisk, vyniká mezi ostatními ranými tištěnými vydáními svou výjimečnou kvalitou provedení.\cite{Davies1996}
    
    \section{Matematická typografie}
    Typografické konvence v matematických vzorcích zajišťují konzistenci v matematických textech a pomáhají čtenářům těchto textů rychle absorbovat nové pojmy.
    
    Matematická notace zahrnuje písmena z různých abeced a také speciální matematické symboly. Písmena v různých fontech mají často v určitých oblastech matematiky konkrétní, ustálený význam.Podrobněji si o tom můžete přečíst zde.\cite{Knuth1979}
    
    \section{Evoluce}
    Typografie je na jedné straně jedním z odvětví grafického designu, na druhé straně kolekce přísných pravidel, která určují použití písma tak, aby bylo pro čtenáře co nejsrozumitelnější text. Během dlouhé historie typografie jsme použili téměř vše, aby naše texty vypadaly krásně a čitelně. Více o tom, jak se typografie v průběhu času měnila, si můžete přečíst $\textup{zde}^{\cite{Forrest2021}}$ a $\textup{zde}^{\cite{Infoamerica2014}}$.
    
    \subsection{Typografie v digitálním designu}
    Hlavní cíl typografie se vůbec nezměnil. Totiž usnadnit život našim čtenářům tím, že se jim bude snadněji číst, co jsme napsali. Když se to dělá dobře, posiluje poselství, které prezentuje.
    Podrobněji o tom, jak byly nastaveny standardy rastrové a vektorové typografie pro osobní systémy a počítače, si můžete přečíst zde.\cite{Charles1983}
    \\
    V dnešní době typografie ovlivňuje i algoritmy. O tom si můžete přečíst zde.\cite{Arnold2020}
    
    \newpage
	\bibliographystyle{czechiso}
	\renewcommand{\refname}{Literatura}
	\bibliography{proj4}
\end{document}
